\documentclass[a4paper, 12pt, ngerman, fleqn]{scrartcl}
\usepackage[utf8]{inputenc}
\usepackage[T1]{fontenc}
\usepackage[ngerman]{babel}
\usepackage{hyperref}
\usepackage{fancyhdr}
\usepackage{titlesec}

\renewcommand{\thesection}{\S \arabic{section}}

\newenvironment{absatz}{ 
 \begin{list}{(\arabic{qcounter})}{\usecounter{qcounter}}
}{
	\end{list}
}

\newcounter{qcounter}


\begin{document}

\title{Geschäftsordnung der GRÜNEN JUGEND Karlsruhe}
\date{}

\maketitle


\section{} 
\begin{absatz}
\item Die Geschäftsordnung regelt den Verlauf der Mitgliederversammlungen und der Aktiventreffen.
\end{absatz}
\section{} 
\begin{absatz}
\item Die Tagesordnung ist einzuhalten, kann aber durch Gaschäftsordungsanträge geändert werden.
\end{absatz}
\section{} 
\begin{absatz}
\item Bei Diskussionen ist ab acht anwesenden Personen oder auf Antrag einer Person eine schriftliche Redeliste zu führen. Auf Antrag einer Person ist die Redeliste hart oder weich zu quotieren. Verständnisfragen mit Zustimmung der
Rednerin und kurze, sich auf den aktuellen Redebeitrag beziehende, Zwischenrufe sind zugelassen. Ansonsten ist die Redeliste einzuhalten.
\item Wird eine weiche Quotierung gefordert, ist wie folgt zu verfahren: Einem männlichen Redebeitrag hat ein weiblicher zu folgen. Möchte sich keine Frau äußern, wird ein weiterer männlicher Beitrag möglich. Menschen, die in einer Diskussion noch keinen Redebeitrag geleistet haben, sind bei der Aufstellung der Redeliste zu bevorzugen. Diese Regelung tritt erst in Kraft, sobald eine Person ein zweites Mal auf der Redeliste auftaucht.
\item Wird eine harte Quotierung gefordert, ist wie folgt zu verfahren: Einem männlichem Redebeitrag hat ein weiblicher zu folgen. Findet sich keine Frau ist die Diskussion zu beenden. Erstredner*innen sind bei der Aufstellung der Redeliste zu bevorzugen.
\end{absatz}
\section{}
\begin{absatz}
\item Auch Nichtmitglieder sind redeberechtigt.
\end{absatz}
\section{} 
\begin{absatz}
\item Bei einer Mitgliederversammlungen können Gaschäftsordungsanträge gestellt werden, sofern diese nicht gegen die Satzung der Grünen Jugend Karlsruhe oder die Satzungen der übergeordneten Verbände verstoßen. Geschäftsordnungsanträge werden durch das Heben beider Arme oder bei körperlichem Unvermögen anderweitig angzeigt. Geschäftsordnungsanträge müssen sofort nach Ende des aktuellen Redebeitrages behandelt werden.
\end{absatz}
\section{}
\begin{absatz}
\item Die Sitzungsleitung wird vom Vorstand vor der Sitzung bestimmt und hat den Auftrag die Einhaltung der Geschäftsordnung zu garantieren. Sie ist auf Geschäftsordnungsantrag neu zu wählen.
\end{absatz}
\section{}
\begin{enumerate}
\item Diese Geschäftsordnung trat nach der MV am 29.05.2012 in Kraft.
\item Sie wurde am 03.09.2013 geändert.
\item Sie wurde am 09.12.2014 geändert.
\item Sie wurde am 20.01.2015 geändert.
\end{enumerate}

\end{document}
